\documentclass[14pt,a4paper]{extarticle}
\usepackage[utf8]{inputenc}
\usepackage[russian]{babel}
\usepackage{geometry}
\usepackage{setspace}
\usepackage{indentfirst}
\usepackage{titlesec}
\usepackage{hyperref}
\usepackage{tocloft}
\usepackage{longtable}
\usepackage{array}
\usepackage{amsmath}

\geometry{left=30mm,right=10mm,top=20mm,bottom=20mm}
\onehalfspacing
\setlength{\parindent}{1.25cm}

\titleformat{\section}{\normalfont\fontsize{14}{16}\bfseries\centering}{\thesection}{1em}{}
\titleformat{\subsection}{\normalfont\fontsize{14}{16}\bfseries\centering}{\thesubsection}{1em}{}

\begin{document}

% ТИТУЛЬНЫЙ ЛИСТ  
\begin{titlepage}
\begin{center}
\textbf{МИНОБРНАУКИ РОССИИ}\\
Федеральное государственное бюджетное образовательное учреждение\\
высшего образования\\
«Гжельский государственный университет»\\
(ГГУ)

\vspace{1cm}
Кафедра «Теории и организации управления»
\vspace{3cm}

\textbf{КУРСОВОЙ ОБЩЕСТВЕННЫЙ ПРОЕКТ ``ОБУЧЕНИЕ СЛУЖЕНИЕМ''}

по дисциплине «Основы проектной деятельности»

\vspace{2cm}

Тема: \textbf{Цифровизация процессов управления волонтёрской деятельностью\\в некоммерческих организациях города Москвы}

\end{center}

\vspace{3cm}

\begin{flushright}
Выполнил: студент 2 курса\\
Института экономики и управления\\
Группа: БИ-БО-23\\
\underline{\hspace{5cm}}\\[10pt]
Проверил: к.э.н., доцент\\
\underline{\hspace{5cm}}
\end{flushright}

\vfill
\begin{center}
Пос. Электроизолятор\\
2026
\end{center}
\end{titlepage}

% ОГЛАВЛЕНИЕ
\newpage
\setcounter{page}{2}
\tableofcontents

% ВВЕДЕНИЕ
\newpage
\section*{Введение}
\addcontentsline{toc}{section}{Введение}

\textbf{Актуальность темы.} В современных условиях развития гражданского общества в Российской Федерации возрастает роль некоммерческих организаций и волонтёрского движения как институтов социальной поддержки населения. По данным Ассоциации волонтёрских центров, в 2025 году в России зарегистрировано более 15 миллионов добровольцев, при этом координация их деятельности часто осуществляется с использованием устаревших методов управления.

% ==== МЕСТО ДЛЯ ЗАПОЛНЕНИЯ СТУДЕНТОМ ====
% Здесь студент может дополнить актуальность своими исследованиями
% ==========

\textbf{Цель курсового проекта} — разработать и частично внедрить информационную систему управления волонтёрской деятельностью для некоммерческих организаций города Москвы.

\textbf{Задачи исследования:}
\begin{enumerate}
\item Провести анализ текущей ситуации в сфере управления волонтёрской деятельностью в НКО города Москвы.
\item Сформулировать проблему, требующую проектного решения.
\item Разработать гипотезу проектного решения.
\item Провести проверку гипотезы путём прототипирования.
\item Разработать паспорт общественного проекта.
\item Реализовать пилотный этап внедрения.
\item Оценить социальный эффект от реализации проекта.
\end{enumerate}

\textbf{Объект исследования} — процессы управления волонтёрской деятельностью в НКО города Москвы.

\textbf{Предмет исследования} — методы и технологии цифровизации процессов координации волонтёрских проектов.

% ГЛАВА 1
\newpage
\section{Анализ ситуации и постановка проблемы}

\subsection{Изучение контекста волонтёрской деятельности в городе Москве}

Для понимания широкого контекста была проведена работа по анализу социальных, экономических, политических и технологических аспектов развития добровольчества в Москве.

% ==== МЕСТО ДЛЯ ЗАПОЛНЕНИЯ СТУДЕНТОМ ====
% Здесь студент описывает социальный контекст
% Экономический контекст
% Политический и правовой контекст
% Технологический контекст
% ==========

\subsection{Идентификация проблемы}

На основе проведённого анализа была сформулирована главная проблема: \textbf{неэффективность управления волонтёрскими ресурсами в НКО города Москвы вследствие отсутствия специализированных информационных систем координации добровольческой деятельности}.

% ==== МЕСТО ДЛЯ ЗАПОЛНЕНИЯ СТУДЕНТОМ ====
% Здесь студент описывает проявления проблемы
% Заинтересованные стороны
% ==========

\subsection{Сбор данных и анализ}

Для более глубокого понимания проблемы был проведён комплексный сбор и анализ данных.

% ==== МЕСТО ДЛЯ ЗАПОЛНЕНИЯ СТУДЕНТОМ ====
% Анкетирование НКО
% Интервьюирование координаторов
% Опрос волонтёров
% Анализ бизнес-процессов
% Статистический анализ
% ==========

\subsection{Взаимодействие с заинтересованными сторонами}

% ==== МЕСТО ДЛЯ ЗАПОЛНЕНИЯ СТУДЕНТОМ ====
% Встречи с руководителями НКО
% Фокус-группы с волонтёрами
% Консультации с экспертами
% Взаимодействие с органами власти
% ==========

\subsection{Уточнение и формулировка проблемы}

Итоговая проблема проекта: \textbf{``Как создать доступную и функциональную информационную систему управления волонтёрской деятельностью для НКО города Москвы?''}

% ГЛАВА 2
\newpage
\section{Выработка гипотезы проектного решения и её проверка}

\subsection{Создание гипотезы}

\textbf{Гипотеза:} ``Разработка и внедрение веб-платформы с мобильным приложением для управления волонтёрской деятельностью позволит НКО города Москвы повысить эффективность использования волонтёрских ресурсов на 30-40\%, сократить административные издержки на 25-35\%, увеличить показатель удержания волонтёров на 20-25\%.''

% ==== МЕСТО ДЛЯ ЗАПОЛНЕНИЯ СТУДЕНТОМ ====
% Обоснование гипотезы
% Критерии проверки
% ==========

\subsection{Планирование эксперимента}

% ==== МЕСТО ДЛЯ ЗАПОЛНЕНИЯ СТУДЕНТОМ ====
% Этапы эксперимента
% Методы сбора данных
% Ресурсы
% ==========

\subsection{Реализация и оценка}

% ==== МЕСТО ДЛЯ ЗАПОЛНЕНИЯ СТУДЕНТОМ ====
% Разработка технического задания
% Разработка прототипа
% Пилотное тестирование
% ==========

\subsection{Анализ и заключение}

% ==== МЕСТО ДЛЯ ЗАПОЛНЕНИЯ СТУДЕНТОМ ====
% Анализ результатов
% Подтверждение/опровержение гипотезы
% ==========

% ГЛАВА 3
\newpage
\section{Разработка и защита паспорта проекта. Реализация общественного проекта}

\subsection{Определение общих целей проекта}

\textbf{Главная цель проекта:} повышение эффективности управления волонтёрской деятельностью в НКО города Москвы посредством создания и внедрения специализированной информационной системы.

% ==== МЕСТО ДЛЯ ЗАПОЛНЕНИЯ СТУДЕНТОМ ====
% Конкретные цели
% Обоснование важности
% ==========

\subsection{Выработка описания проекта}

% ==== МЕСТО ДЛЯ ЗАПОЛНЕНИЯ СТУДЕНТОМ ====
% Что планируется сделать
% Как будет осуществлено
% Какие ресурсы будут использованы
% ==========

\subsection{Определение задач и плана работы}

\begin{longtable}{|p{1cm}|p{5cm}|p{3cm}|p{3cm}|p{2cm}|}
\caption{План-график реализации проекта} \\
\hline
\textbf{№} & \textbf{Задача} & \textbf{Срок} & \textbf{Ответственный} & \textbf{Результат} \\
\hline
\endfirsthead
\multicolumn{5}{c}{\tablename\ \thetable\ -- продолжение} \\
\hline
\textbf{№} & \textbf{Задача} & \textbf{Срок} & \textbf{Ответственный} & \textbf{Результат} \\
\hline
\endhead
\hline
\endfoot

% ==== МЕСТО ДЛЯ ЗАПОЛНЕНИЯ СТУДЕНТОМ ====
1 & Формирование команды & Май 2026 & Руководитель & Команда сформирована \\
\hline
% ... студент добавляет остальные задачи ...
% ==========

\end{longtable}

\subsection{Оценка необходимых ресурсов}

% ==== МЕСТО ДЛЯ ЗАПОЛНЕНИЯ СТУДЕНТОМ ====
% Человеческие ресурсы
% Финансовые средства
% Оборудование
% Технологические ресурсы
% ==========

\subsection{Защита паспорта проекта}

% ==== МЕСТО ДЛЯ ЗАПОЛНЕНИЯ СТУДЕНТОМ ====
% Процедура защиты
% Результаты защиты
% Обратная связь и рекомендации
% ==========

\subsection{Реализация общественного проекта}

\subsubsection{Прототипирование}

% ==== МЕСТО ДЛЯ ЗАПОЛНЕНИЯ СТУДЕНТОМ ====
% Описание процесса прототипирования
% ==========

\subsubsection{Разработка и реализация}

% ==== МЕСТО ДЛЯ ЗАПОЛНЕНИЯ СТУДЕНТОМ ====
% Описание процесса разработки
% ==========

\subsubsection{Тестирование и улучшение}

% ==== МЕСТО ДЛЯ ЗАПОЛНЕНИЯ СТУДЕНТОМ ====
% Описание тестирования
% ==========

\subsubsection{Оценка результатов}

% ==== МЕСТО ДЛЯ ЗАПОЛНЕНИЯ СТУДЕНТОМ ====
% Оценка достигнутых результатов
% ==========

% ЗАКЛЮЧЕНИЕ
\newpage
\section*{Заключение}
\addcontentsline{toc}{section}{Заключение}

Выполнение курсового общественного проекта позволило достичь поставленных целей и решить определённые задачи.

% ==== МЕСТО ДЛЯ ЗАПОЛНЕНИЯ СТУДЕНТОМ ====
% Основные результаты
% Социальная значимость
% Применение профессиональных компетенций
% Образовательные результаты
% Перспективы развития
% ==========

% СПИСОК ЛИТЕРАТУРЫ
\newpage
\section*{Список используемых источников и литературы}
\addcontentsline{toc}{section}{Список используемых источников и литературы}

\textbf{Нормативные правовые акты:}

1. Конституция Российской Федерации (принята всенародным голосованием 12.12.1993).

2. Федеральный закон от 11.08.1995 № 135-ФЗ ``О благотворительной деятельности и добровольчестве (волонтёрстве)''.

% ==== МЕСТО ДЛЯ ЗАПОЛНЕНИЯ СТУДЕНТОМ ====
% Студент добавляет остальные источники
% ==========

\textbf{Учебники, монографии, статьи:}

% ==== МЕСТО ДЛЯ ЗАПОЛНЕНИЯ СТУДЕНТОМ ====
% Студент добавляет источники
% ==========

\textbf{Интернет-ресурсы:}

% ==== МЕСТО ДЛЯ ЗАПОЛНЕНИЯ СТУДЕНТОМ ====
% Студент добавляет источники
% ==========

% ПРИЛОЖЕНИЯ
\newpage
\section*{Приложения}
\addcontentsline{toc}{section}{Приложения}

\begin{center}
\textbf{Приложение 1}

\textbf{Паспорт общественного проекта}
\end{center}

\begin{longtable}{|p{4cm}|p{10cm}|}
\hline
\textbf{Характеристика} & \textbf{Описание} \\
\hline
Наименование проекта & Цифровизация процессов управления волонтёрской деятельностью в НКО города Москвы \\
\hline
Разработчик проекта & \underline{\hspace{8cm}} \\
\hline
Координатор проекта & \underline{\hspace{8cm}} \\
\hline
Партнёрские организации & \underline{\hspace{8cm}} \\
& \underline{\hspace{8cm}} \\
\hline
Целевая аудитория & НКО города Москвы, координаторы волонтёров, добровольцы \\
\hline
Цель проекта & Повышение эффективности управления волонтёрской деятельностью \\
\hline
Задачи проекта & % ==== МЕСТО ДЛЯ ЗАПОЛНЕНИЯ СТУДЕНТОМ ==== \\
\hline
Сроки реализации & \underline{\hspace{8cm}} \\
\hline
Ресурсное обеспечение & % ==== МЕСТО ДЛЯ ЗАПОЛНЕНИЯ СТУДЕНТОМ ==== \\
\hline
Ключевые мероприятия & % ==== МЕСТО ДЛЯ ЗАПОЛНЕНИЯ СТУДЕНТОМ ==== \\
\hline
Ожидаемые результаты & % ==== МЕСТО ДЛЯ ЗАПОЛНЕНИЯ СТУДЕНТОМ ==== \\
\hline
Критерии эффективности & % ==== МЕСТО ДЛЯ ЗАПОЛНЕНИЯ СТУДЕНТОМ ==== \\
\hline
\end{longtable}

\end{document}
